%--------- HAF 5-10-2015--------------------
\documentclass[a4paper,thmohb,12pt,logo]{mohamed2}
% si pas logo
%\documentclass[twocolumn,letterpaper,thmohb,11pt]{mohamed2}
%%%%%%%%%%%%%%%%%%%%%%
\usepackage{mohamed2}
\usepackage[OT1]{fontenc}
\usepackage[ansinew]{inputenc}
\usepackage[french]{babel}
\usepackage{pgfplots}
\usepackage{tkz-fct}
\usetikzlibrary{intersections}
\usepackage{float}
\usepackage{hyphenat}
\usepackage{siunitx}
\sisetup{per=slash}
\usepackage{textcomp}
\usepackage{amsmath}
\usepackage{amssymb}
\usepackage{multicol}
\usepackage{lastpage}
\usepackage{siunitx}
\usepackage{listings}
\usepackage{booktabs}
\usepackage[tikz]{bclogo}
\usepackage{pstricks}
\usepackage[dvips,ps2pdf,frenchlinks=true,breaklinks=true]{hyperref}
\hypersetup{pdfauthor={Mohamed HAFIANE},pdftitle={TP ; L3 Hydraulique},pdfsubject={Hydro-Informatique},pdfkeywords={LaTeX, PsTricks,Hyperref,...}}
%%%%%%%%%%%%%%%%%%%%%%%%%%%%%%%%%%%%%%%%
\usepackage{listings}
%%%%%%%%%%%%%%%%
\input{map2}
\input{HafMacros}
\input{HAF-Hyphenation}
\input{HAF-Langages}
\graphicspath{{Images/}}
%%%%%%%%%%%%%%%%%%%%%%%%%%%%%%%%
\definecolor{HAF1}{RGB}{255,235,205} %pour Algorithme
%%%%%%%%%%%%%%%%%%%%%%%%%%%%%%%%
\newlength{\len}
\newlength{\plotwidth}
\newlength{\plotheight}
\newcommand{\getvalue}[1]{\pgfkeysvalueof{/pgfplots/#1}}

%output will be given by \pgfmathresult
\newcommand{\xcoord}[1]% #1 = node name
{\pgfplotsextra{%
		\pgfextractx{\len}{\pgfpointdiff{\pgfplotspointaxisxy{0}{0}}{\pgfpointanchor{#1}{center}}}%
		\pgfextractx{\plotwidth}{\pgfpointdiff{\pgfplotspointaxisxy{\getvalue{xmin}}{0}}%
			{\pgfplotspointaxisxy{\getvalue{xmax}}{0}}}%
		\pgfmathparse{\len*(\getvalue{xmax}-\getvalue{xmin})/\plotwidth}%
}}
\newcommand{\ycoord}[1]% #1 = node name
{\pgfplotsextra{%
		\pgfextracty{\len}{\pgfpointdiff{\pgfplotspointaxisxy{0}{0}}{\pgfpointanchor{#1}{center}}}%
		\pgfextracty{\plotheight}{\pgfpointdiff{\pgfplotspointaxisxy{0}{\getvalue{ymin}}}%
			{\pgfplotspointaxisxy{0}{\getvalue{ymax}}}}%
		\pgfmathparse{\len*(\getvalue{ymax}-\getvalue{ymin})/\plotheight}%
}}
%%%%%%%%%%%%%%%%%%%%%%%%%%%%%%%%
\usepackage[a4paper,textwidth=18cm,textheight=26.5cm,centering]{geometry}
\usepackage[most]{tcolorbox}
% some colors
\definecolor{bistre}{rgb}{.75,.50,.30}
\definecolor{Maroon}{rgb}{0.5,0.0,0.0}
\definecolor{fondpaille}{cmyk}{0,0,0.1,0}
\definecolor{HAF1}{RGB}{255,235,205}
\definecolor{light-gray}{gray}{0.95}
\usepackage{empheq}
\usepackage{tcolorbox}
\uselanguage{French}
\languagepath{French}
% put color to \boxed math command
\newcommand*{\boxcolor}{red}
\makeatletter
\renewcommand{\boxed}[1]{\textcolor{\boxcolor}{%
		\tikz[baseline={([yshift=-1ex]current bounding box.center)}] \node [rectangle, minimum width=1ex,rounded corners,draw,fill=HAF1,thick] {\normalcolor\m@th$\displaystyle#1$};}}
\begin{document}
	\universite{UNIVERSIT� DE SA�DA}
	\institut{FACULT� DE TECHNOLOGIE \\
		D�PARTEMENT DE G�NIE CIVIL ET D'HYDRAULIQUE}
	\author{M.~\textsc{HAFIANE}}
	\module{M�thodes num�riques (S4, 2\ieme Ann�e licence "Hydraulique \& G�nie civil")} 
	\exam{TD}
	\theyear{2019-2020}
	\date{}
	\thetime{}
	\title{}
	\maketitle
\vfill
\begin{minipage}{0.75\textwidth}
\textbf{\underline{Solution num�rique}} :
Toutes les m�thodes utilisent la discr�tisation de l'une des formes suivantes de l'�quation GVF : ($ S_0 $ la pente du fond du canal et $ S_f $ la pente de la ligne de chage totale )
\newtcolorbox{mymathbox}[1][]{colback=white, sharp corners, #1}
\newtcbox{\othermathbox}[1][]{nobeforeafter, math upper, tcbox raise base, enhanced, sharp corners, colback=blue!10, colframe=red!30!black, 
	drop fuzzy shadow, left=1em, top=1em, right=1em, bottom=1em}
\begin{empheq}[box=\othermathbox]{align}
	\frac{\mathrm{d}H}{\mathrm{d}x}&=-S_f&\hspace*{1.5cm}(\text{Charge totale : } H=z+y+\dfrac{v^2}{2g})\\
\frac{\mathrm{d}E}{\mathrm{d}x}&=S_o-S_f&\hspace*{1.5cm}(\text{Energie sp�cifique : } E=y+\dfrac{v^2}{2g})\\
\frac{\mathrm{d}y}{\mathrm{d}x}&=\dfrac{S_o-S_f}{1-\mathrm{Fr}^2}&\hspace*{1.5cm}(\text{Profondeur } y)
\end{empheq} 
\end{minipage}	

\vfill
\end{document}
